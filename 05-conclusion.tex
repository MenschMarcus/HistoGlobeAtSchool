\section{Conclusion}
Platzhalter\\

\subsection{Study Results}
\subsection{Discussion}
\subsection{Future Work} 
\subsubsection{Dynamics and animated transitions} % (fold)
\label{sub:dynamics_and_animated_transitions}
While testing HistoGlobe we found out that the pupils wanted a better visualization of historical connectivity, in the form of  more dynamic display of the data with better respect to the context of historical events. We think one should try to show the flow of history with more detail and animation. 



% subsection dynamics_and_animated_transitions (end)

\subsubsection{Content Generation Backend} % (fold)
\label{sub:content_generation_backend}

For our studies we used a self generated, wikipedia.org based database, which was sufficient for our first study. Nevertheless it would be great if a teacher or students could compile lessons and homework with a tool. This tool should be able to create, modify and delete hivents easily, and to add them in arbitrary form to categories.
Another helpful functionality would be the automatic generation of a hivent from a given wikipedia article. 
% subsection content_generation_backend (end)


\subsubsection{Country Generation Backend} % (fold)
\label{sub:country_generation_backend}
Platzhalter
% subsection country_generation_backend (end)

\subsubsection{Mobile Version} % (fold)
\label{sub:mobile_version}
HistoGlobe was designed, on the UI and software level, to be usable on tablets and phones. But at this point, optimization is sub-optimal.
Further tuning to style and usability are necessary.
% subsection mobile_version (end)



% Für Bilder (zentriert):
%\begin{center}
	%\includegraphics[width=0.8\textwidth]{images/Bildname.png}
%end{center}



\newpage
