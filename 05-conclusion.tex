\section{Conclusion} % (fold)
\label{sec:conclusion}

\paragraph{User Centered Design} % (fold)
We have implemented a visualization which is usable for the teacher and understandable for the students. We also had valuable feedback throughout the entire development process, which was the basis for the good result of the final field study. We did not strictly follow the ideas of User Centered Design, because we hardly used paper prototypes or think aloud protocols, but implemented most of the design ideas directly to show them to the teacher. Since we threw out some features, there was a waste of time. A big problem of the design approach is that we developed the software based on the ideas of only one teacher, so we can not reason on how applicable our solution is to most schools. We also did not show the visualization to students before the final field study.

\paragraph{Visualisation} % (fold)
Our visualization can be navigated via Map and Timeline. Historical events can be found through the Hivent List, the Search Bar, or directly on the Map or the Topic Bars represented as markers with supported linking. We are able to visualize area and border changes and to show themes. Our high contrast and minimum UI mode allows the users to customize their view on HistoGlobe. Therefore we are able to visualize \textit{what}, \textit{where}, \textit{when} and \textit{how} happened. But we are not yet able to show \textit{why} something happened -- the visualization of coherences in history was one main goal of the project which we have not achieved.

\paragraph{Cooperation with Designer} % (fold)
In the User Centered Design approach and for the Interface Design we collaborated with a designer from the Media Arts and Design program, Tobias Westphal. The collaboration was very productive and successful for both sides.Together we developed a concept that is visually appealing, easy to use and realizable for the programmers. He created great designs and it it was not for him the meetings with the teacher would have not been so productive. The only negative aspect to name is that this was no common project between Media Art and Design and Computer Science and Media.

\paragraph{Organisation} % (fold)
We have worked with Git which was very productive. We also had a Masterplan, so everyone of us could see what to do next and which priority the particular feature to implement had. We learned how to collaboratively work on a project and solve problems together.

\paragraph{Goal for the semester} % (fold)
Finally we created a visualization that both teachers and students were able to use without any problems. We are proud that the students understood the goal of \HG and were able to solve their task in the second field study. Because of the successful field study we can conclude that we reached our goal for the semester.

\subsection{Future Work} % (fold)
\label{sub:future_work}

\paragraph{Dynamics and animated transitions} % (fold)
\label{sub:dynamics_and_animated_transitions}
While testing \HG we found out that the students wanted a better visualization of historical connectivity, in the form of  more dynamic display of the data with better respect to the context of historical events. We think one should try to show the flow of history with more detail and animation.

% subsection dynamics_and_animated_transitions (end)

\paragraph{Content Generation Backend} % (fold)
\label{sub:content_generation_backend}

For our studies we used a self generated, wikipedia.org based database, which was sufficient for our first study. Nevertheless it would be great if a teacher or students could compile lessons and homework with a tool. This tool should be able to create, modify and delete Hivents easily, and to add them in arbitrary form to categories.
Another helpful functionality would be the automatic generation of a Hivent from a given wikipedia article.
% subsection content_generation_backend (end)

\paragraph{Country Generation Backend} % (fold)
\label{sub:country_generation_backend}
For the future, an editor for storing, managing and analyzing historic changes would be desirable, because the data acquisition part took a large share of the projects time. This task includes the improvement of the data model as described in section \ref{sub:map}.
% subsection country_generation_backend (end)

\paragraph{Mobile Version} % (fold)
\label{sub:mobile_version}
\HG was designed, on the UI and software level, to be usable on tablets and phones. But at this point, optimization is sub-optimal.
Further tuning to style and usability are necessary.
% subsection mobile_version (end)

% subsection future_work (end)
% section conclusion (end)
