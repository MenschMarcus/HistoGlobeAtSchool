\section{Conclusion} % (fold)
\label{sec:conclusion}

\paragraph{User Centered Design} % (fold)
We have implemented a visualisation, which is usable for the teacher and understandable for the students. We also had valuable feedback throughout the entire development process, but we did not strictly apllied the user centered design, which means that we had no prototypes or think aloud protocols. A negative aspect to name ist that we had only one teacher and only a few studets to test our implementation with.

\paragraph{Visualisation} % (fold)
Our visualisation can be navigated via Map and Timeline. Historical events can be found through the Hivent List, the Search Bar, or directly on the Map or the Topic Bars represented as markers with supported linking. We are able to visualize areas and border changes and to show themes. Our high contrast and minimum UI mode allows the users to customize their view on HistoGlobe. Therefore we are able to visualize \textit{what}, \textit{where}, \textit{when} and \textit{how} happend. But we are not yet able to show \textit{why} something happend, so the visualisation of coherences is not developed.

\paragraph{Cooperation with Designer} % (fold)
To the work with the designer has to be pointed out that it was very collaborative and very productive. He provided us with greatly designs. The only negative aspect to name is that this was no common project. %between Media Art & Design and Computer Science and Media

\paragraph{Organisation} % (fold)
We have worked with git, which was very productive. We also had a masterplan, so everyone of us could see what to do next and which priority the particular feature to implement had.

\paragraph{Goal for the semester} % (fold)
The teacher was able to use visualization and the students understood the goal of HistoGlobe. Our field study was sucessful and so we are proud to say that we reached our goal for the semester.

\subsection{Future Work} % (fold)
\label{sub:future_work}

\subsubsection{Dynamics and animated transitions} % (fold)
\label{sub:dynamics_and_animated_transitions}
While testing HistoGlobe we found out that the pupils wanted a better visualization of historical connectivity, in the form of  more dynamic display of the data with better respect to the context of historical events. We think one should try to show the flow of history with more detail and animation.


% subsection dynamics_and_animated_transitions (end)

\subsubsection{Content Generation Backend} % (fold)
\label{sub:content_generation_backend}

For our studies we used a self generated, wikipedia.org based database, which was sufficient for our first study. Nevertheless it would be great if a teacher or students could compile lessons and homework with a tool. This tool should be able to create, modify and delete hivents easily, and to add them in arbitrary form to categories.
Another helpful functionality would be the automatic generation of a Hivent from a given wikipedia article.
% subsection content_generation_backend (end)

\subsubsection{Country Generation Backend} % (fold)
\label{sub:country_generation_backend}
For the future, an editor for storing, managing and analyzing historic changes would be desireable, because the data aqcuisiton part took a large share of the projects time. This task includes the improvement of the data model as described in section \ref{sub:map}.
% subsection country_generation_backend (end)

\subsubsection{Mobile Version} % (fold)
\label{sub:mobile_version}
HistoGlobe was designed, on the UI and software level, to be usable on tablets and phones. But at this point, optimization is sub-optimal.
Further tuning to style and usability are necessary.
% subsection mobile_version (end)

% subsection future_work (end)
% section conclusion (end)
