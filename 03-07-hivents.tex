\subsection{Hivents} % (fold)
\label{sec:hivents}
\subsubsection{General} % (fold)
\label{sub:general}
Hivents mean "Historical Events", and are the way historical events are defined and presented. They are stored in a database, and have attributes such as name, location, start- and endyear, description and asocciated media.
They are one of the most important concepts to vizualze history in HistoGlobe.

% subsection general (end)
\subsubsection{Behaviour} % (fold)
\label{sub:behaviour}
Hivents are represented on several locations in the UI, with the map, the hivent list and the timeline beeing the most important ones.
Since usually more than one hivent is beeing shown, it's important to signify the representation between one hivent in different UI-Elements.
The main interface events to which the feedback occures are mouse hovering and clicking, which leads to the hivent beeing highlighted in all ui-elements.

todo: Bild

Upon beeng clicked on, it changes its status to active. An active hivent gets focused in the map, it's marker are hightlighted, it gets tagged in the url bar and the hivent box opens.

\subsubsection{Labels on Map}
Hivents on the Map were only represented by a marker, which automatically clustered when they are close together.
The teacher demanded more information, so we attached labels to the hivent markers.
In the first implementation the hivents name was shown, but we realized this isn't appropriate on a map, so we switched to the location.

todo: Besser schreiben

If they overlap each other, we implemented an easy algorithm to adjust them. 
This is done by checking for an overlap, and then moving the left label to the left.
On high zoom levels the amount of labels was so high the map wasn't readable anymore, so we remove them from a certain zoom level on.


\subsubsection{Hivent Regions}
A lot of historical events took place over a region, such as wars, so we wanted a region representation of hivents on the map.
We implemented an additional type of map marker to do this.
We added an additional optianal attriubute to the hivents, containing the polygon representing the region.

todo: Bild

We decided to not show them to the teachers, because the polygons which define the regions are crude, and don't look good.



% subsection hghghg (end)
% section section_name (end)

% subsection behaviour (end)
% section hivents (end)
